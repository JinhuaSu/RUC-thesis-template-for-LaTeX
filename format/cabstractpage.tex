% the abstract
\begin{abstractzh}
气候变化下频发的气象灾害直接影响农业产量,“投产-收产-供需”的复杂关系影响农业信贷规模,进而影响农业金融稳定性。金融稳定需要积极应对气候风险,气候的变化以及气象灾害有征兆、可分析,对传导路径的分析与解释有助于充分利用气象数据,提前应对可能的农业金融风险。现有研究已经初步构建气候影响农业金融的理论框架、并在宏观气象数据、大周期的维度进行实证分析,站在巨人的肩膀,本研究有机会深入探究微观气象数据构成的风险因子的差异和传导路径,进一步验证和补充现有研究的结论。在对江苏各县数据的实证分析中,我们总结了风险传导的特点,并加入了区域性分析,得到了1.xxx2.xxx3.xxx的具体结论,为江苏省农业金融如何应当气候风险提供了探索性思路。
相对于宏观的是微观,如何处理各县GDP,如何解释脉冲,如何附录,如何保证文章有内容
\end{abstractzh}
