% the abstract
\begin{abstracten}
The frequent meteorological disasters under climate change directly affect agricultural output. The complex relationship of "investment-production-supply-demand" impacts the scale of agricultural credit and, consequently, the stability of agricultural finance. Financial stability requires active responses to climate risks. Changes in climate and meteorological disasters are foreseeable and analyzable. Analyzing and interpreting the transmission pathways helps to make full use of meteorological data to proactively address potential agricultural financial risks. Existing research has preliminarily established a theoretical framework for the impact of climate on agricultural finance and conducted empirical analysis in terms of macro meteorological data and large cycles. Standing on the shoulders of giants, this study has the opportunity to delve into the differences and transmission pathways of risk factors constituted by micro meteorological data, further verifying and supplementing the conclusions of existing research. In the empirical analysis of data from various counties in Jiangsu, we have summarized the characteristics of risk transmission and incorporated regional analysis, arriving at specific conclusions of 1.xxx 2.xxx 3.xxx, which provides exploratory ideas on how agricultural finance in Jiangsu Province should respond to climate risks.
\end{abstracten}
